\documentclass[a4paper]{article} 

\usepackage{epsfig}
\usepackage{color}
\usepackage{graphicx}

\definecolor{orange}{rgb}{1,0.5,0}
\definecolor{lila}{rgb}{0.8,0,0.5}
\definecolor{lightblue}{rgb}{0,0.5,1}
\definecolor{darkgreen}{rgb}{0,0.6,0.3}

\usepackage{amsmath,amsfonts,amssymb} % use AMS material
\usepackage{setspace} %\doublespacing

\newcommand{\la}[1]{}
\newcommand{\co}[1]{{\em{\color{orange}#1}}}
\newcommand{\coout}[1]{}

\renewcommand{\familydefault}{\sfdefault}

\newcommand{\tab}[1]{{\texttt{\color{red} #1}}}
\newcommand{\col}[1]{\texttt{\color{blue} #1}}
\newcommand{\defext}[1] {\texttt{\color{lightblue} #1}}
\newcommand{\defint}[1] {\texttt{\color{darkgreen} #1}}
\newcommand{\nick}[1] {\texttt{\color{lila} #1}}

\input{mymathdefinitions.tex}
\input{mysettings.tex}

\renewcommand{\headsep}{-1cm}
\renewcommand{\textheight}{25cm}


% ------------------------------------------------------------------

\begin{document}

\title{SBtab: Standardised data tables  for Systems Biology}

\author{Wolfram Liebermeister, Timo Lubitz and Jens Hahn}
\date{}

\maketitle

\begin{abstract}
  Data tables in the form of spreadsheet or delimited text files are
  the most popular data format in Systems Biology. However, they are
  often not sufficiently structured and they lack clear naming
  conventions that are required for modelling.  The SBtab format is an
  attempt to establish an easy-to-use format that is both flexible and
  clearly structured.  It comprises defined table types for different
  kinds of data; syntax rules for usage of names, shortnames, and
  database identifiers used for annotation; 
  and standardised formulae for reaction stoichiometries. 
  Furthermore, SBtab enables the user to easily define and use own table types to adjust
  SBtab to different types of information.
  Predefined table types can be used to define biochemical network
  models and the biochemical constants therein. 
\end{abstract}

\section{Introduction} 

Spreadsheets and delimited text tables are the most popular data
formats for biological data.  They are easy to use and can hold
various types of data.  Tables can not only store omics data, but also
metabolic network models described by lists of biochemical reactions.
However, when tables are exchanged within scientific collaborations
and used as a basis for modelling, modellers usually prefer tables
that can be processed automatically, and the flexibility of
spreadsheets can become a disadvantage.  If table structures and
nomenclature vary from case to case, tables become difficult to parse
and each file may require a new parser. Furthermore, different naming
conventions -- for instance, for biochemical compounds -- make it
hard to combine data, for instance a metabolic network model and omics
data set produced by different researchers.  Therefore, rules for
structuring tables and for consistent naming and annotations can make
tables much more useful as exchange formats in Systems Biology
collaborations and for usage in software tools.  In the following, we
propose a set of conventions for data tables, called the SBtab format,
that are supposed to make tables easier and safer to work with.  We
start with a couple of use cases and then continue with a more formal
specification of the format.

\paragraph{Use case 1: A stoichiometric metabolic model} 
A stoichiometric metabolic model can be defined by a list of
biochemical reaction formulae, specifying the substrates, products,
and their stoichiometric coefficients.  Such reactions can be listed
in a single column of a spreadsheet. In practice, however, additional
information may be provided: each reaction can have a number or
identifier (defined only within the model) and can be linked to an
entry in the database KEGG Reaction \cite{KEGG}.  Furthermore,
reactions may be catalysed by enzymes and therefore be related to
certain genes.  All information could be stored in the following
table:

\begin{center} {\tt {\small
      \begin{tabular}{|l|l|l|l|}
        \hline
        Reaction & Sum formula & KEGG ID & Gene\\ \hline
        R1 & ATP  + F6P <=> ADP + F16P &R00658 & pfk    \\ \hline
        R2 & F16P + H2O <=> F6P + Pi   &R01015 & fbp   \\ \hline
      \end{tabular}}\ \\[5mm]
  }
\end{center}

where \texttt{ATP}, \texttt{F6P}, \texttt{ADP}, \texttt{F16P},
\texttt{H2O}, and \texttt{Pi} are shortnames for metabolites to be
used in the model.  Although the information is complete and
unambigous, the parser still has to recognise that the columns
\texttt{Sum formula} and \texttt{KEGG ID} contain reaction formulae
and identifiers in certain formats. If the column names and the syntax
of the reaction formulae vary from table to table (e.g., \texttt{<->}
is used instead of \texttt{<=>}), parsing becomes tedious.  In the
SBtab format, the table would look a little more complicated, but is
easy to parse automatically:

\begin{center} {\tt \small {
      \begin{tabular}{|l|l|l|l|}
        \hline 
        \tab{!!SBtab} & \tab{TableName='Reaction'}& \tab{TableType='Reaction'}&\\
        \hline\hline
        \col{!Reaction} & \col{!SumFormula} &\col{!MiriamID}::\defext{urn:miriam:kegg.reaction} &   \col{!GeneName}\\ \hline
        \nick{R1} & \nick{ATP} + \nick{F6P} <=> \nick{ADP} + \nick{F16P} &R00658                 &   pfk \\ \hline
        \nick{R2} & \nick{F16P} + \nick{H2O} <=> \nick{F6P} + \nick{Pi}  &R01015                 &   fbp \\ \hline
      \end{tabular}}\ \\[5mm]
  }
\end{center}

Different types of information are highlighted by colours (which are
not part of the SBtab format). The SBtab table differs from the
original table in several ways: the first line (starting with
\texttt{!!})  declares that the table is an SBtab table of the type
\tab{Reaction} and therefore supposed to satisfy syntactic rules for
this table type. The following line contains the column headers.  They
start with the \texttt{!} character (and are highlighted here in
blue), emphasising that they were not chosen ad hoc by the user, but
stem from a controlled vocabulary. The predefined column headers do
not contain whitespaces.  The header \texttt{KEGG ID} has been
replaced by the term
\col{!MiriamID}\texttt{::}\defext{urn:miriam:kegg.reaction}.  This
looks complicated, but it directly allows parsers to use a web service
(the MIRIAM resources \cite{laibe2007miriam}) to look up the KEGG
database, and since Miriam IDs are guaranteed to remain stable in
time, this will even work if the KEGG database itself changes its URL
address on the internet.  The expression
\defext{urn:miriam:kegg.reaction} is defined by the MIRIAM resources
and used within SBtab.  The syntax of the reaction formulae is also
uniquely defined. In particular, the shortnames of metabolites (marked
in magenta) must not contain any whitespaces or special characters,
which again simplifies parsing and maybe as variable names in software
tools.  The meaning of these shortnames has to be defined by the user.
This can be achieved by providing standard names or database
identifiers in a second table of type \tab{Compound}:

\begin{center} {\tt \small
    \begin{tabular}{|l|l|l|}
\hline
    \tab{!!SBtab} & \tab{TableName='Compound'} & \tab{TableType='Compound'} \\
      \hline\hline
      \col{!Compound}&\col{!Name} & \col{!MiriamID}::\defext{urn:miriam:kegg.compound}\\
      \hline
      \nick{F6P} & Fructose-6-phosphate      & C05345 \\ \hline
      \nick{ATP} & ATP                       & C00002 \\ \hline
      \nick{ADP} & ADP                       & C00008 \\ \hline
      \nick{F16P}& Fructose-1,6-bisphosphate & C00354 \\ \hline
      \nick{H2O} & Water                     & C00001 \\ \hline
      \nick{Pi}  & Inorganic phosphate       & C00009 \\ \hline
      \nick{PEP} & Phosphoenolpyruvate       & C00074 \\ \hline
      \nick{AMP} & AMP                       & C00020 \\
      \hline
\end{tabular}
}
\end{center}

Both tables together form a document describing a model.  In practice,
they can be stored as separate files, as separate sheets in a
spreadsheet file, or as parts of a single table. The following example
contains all necessary information to build a structural metabolic
model in the SBML (Systems Biology Markup Language) format
\cite{hfsb:03}:

{\tt \small
\begin{center}
\begin{tabular}{|l|l|l|l|l|}
\hline
\tab{!!SBtab} & \tab{TableName='Reaction'} & \tab{TableType='Reaction'} & \\
\hline\hline
\col{!Reaction} & \col{!SumFormula}            &\col{!MiriamID}::\defext{urn:miriam:kegg.reaction} &\col{!SBML:reaction:id} \\ \hline 
\nick{R1}       & \nick{ATP} + \nick{F6P} <=> \nick{ADP}  + \nick{F16P} &R00658  &r1       \\ \hline
\nick{R2}       & \nick{F16P} + \nick{H2O} <=> \nick{F6P} + \nick{Pi}  &R01015 &r2        \\ \hline\hline
\tab{!!SBtab} & \tab{TableName='Compound'} & \tab{TableType='Compound'} & \\
\hline\hline
\col{!Compound}&\col{!Name} & \col{!MiriamID}::\defext{urn:miriam:kegg.compound}&\col{!SBML:species:id}\\
\hline
\nick{F6P} & Fructose-6-phosphate      & C05345 &f6p \\ \hline
\nick{ATP} & ATP                       & C00002 &atp \\ \hline
\nick{ADP} & ADP                       & C00008 &adp \\ \hline
... & ... & ...                       & ... \\
\hline
\end{tabular}
\end{center}
}

Here, we have added new identifiers (in the columns \col{SBML:reaction:id}
and \col{SBML:species:id}) that replace the entries of \tab{Reaction}
and \tab{Compound} when the tables are translated into SBML. This is
necessary if the original shortnames do not comply with SBML's rules
for element identifiers.


\paragraph{Use case 2: Kinetic constants} In a second example, we
specify numerical parameters, for example the kinetic constants and
metabolite concentrations that appear in a kinetic model. Each
quantity can be related to a compound (e.g., a concentration), to a
reaction (e.g., an equilibrium constant), or to several biological
items (e.g., an enzyme and a compound, in the case of Michaelis-Menten
constants). As in the previous example, these items can be specified
by unique identifiers, e.g., KEGG compound or reaction identifiers.
Furthermore, each quantity has a value and a physical unit.  In the
SBtab format, we arrange this information in a table of type
\tab{Quantity}. Each row contains all  information about
one of the quantities:

{\tt \tiny
  \begin{center}
    \begin{tabular}{|l|l|l|l|l|l|l|l|l|l|l|l|l|l|l|ll}
      \hline
      \tab{!!SBtab} & \tab{TableName='Quantity'} & \tab{TableType='Quantity'} & & & \\\hline\hline
      \col{!Quantity} &   \col{!QuantityType} &  \col{!MiriamID:Reaction::}\defext{urn:miriam:kegg.reaction} & \col{!MiriamID:Compound::}\defext{urn:miriam:kegg.compound} &\defint{!Value} &\col{!Unit} \\ \hline
      \nick{keq\_R1}& \defext{equilibrium constant}      &  R01061 &        & 0.156 & \defext{1 }\\ \hline
      \nick{kmc\_R1\_C1}& \defext{Michaelis constant}        &  R01061 & C00003 & 0.96  & \defext{mM}\\ \hline
      \nick{kic\_R1\_C1}& \defext{inhibition constant}       &  R01070 & C00111 & 0.13  & \defext{mM}\\ \hline
      \nick{con\_C1}& \defext{concentration}             &         & C00118 & 0.203 & \defext{mM}\\ \hline
      ... &   ... &   ... &   ... &   ... &  ...  \\ \hline
    \end{tabular}
  \end{center}
}

The first two columns specify a name and a type for each quantity.
The quantity types (\defext{substrate catalytic rate constant},
\defext{equilibrium constant} etc, marked in cyan) are not chosen
ad hoc, but stem from the Systems Biology Ontology (SBO). This ensures
a unique spelling and allows software to look up definitions and
further information at the SBO web services. The biological items (in
this case, reactions, compounds, or both) are specified in the
following two columns. Unnecessary fields remain empty. The column
name \defint{Value} is defined for SBtab -- like some other
mathematical terms -- and is therefore marked in light blue (arbitrary
values in this example \coout{use realistic values!}). Units names
are defined like in SBML (see below).  If the table is used together
with a metabolic model, we can compound and reaction identifiers from
the model instead of the MIRIAM annotations. In this case, the table
would look like this:

\begin{center} {\tt \small
    \begin{tabular}{|l|l|l|l|l|l|l|l|l|l|l|l|l|l|ll}
      \hline 
      \tab{!!SBtab} & \tab{TableName='Quantity'} & \tab{TableType='Quantity'} & & \\
      \hline\hline
      \col{!QuantityType} &  \col{!SBML:reaction:id} & \col{!SBML:species:id} &\defint{!Value} &\col{!Unit} \\ \hline
      \defext{equilibrium constant}      &  r1 &        & 0.156 & \defext{1 }\\ \hline
      \defext{Michaelis constant}        &  r1 & atp & 0.96  & \defext{mM}\\ \hline
      \defext{inhibition constant}       &  r1 & atp & 0.13  & \defext{mM}\\ \hline
      \defext{concentration}             &         & atp & 1.5 & \defext{mM}\\ \hline
  ... &   ... &   ... &   ... & ...   \\ \hline
\end{tabular}
}
\end{center}

This table, together with a stoichiometric model and a choice of
standard rate laws (like the modular rate laws \cite{liuk:10})
completely defines a kinetic metabolic model.


\section{The SBtab format}
\label{chapter2}

\subsection{Overview}

SBtab comprises a list of conventions about the structure,
nomenclature, syntax, and annotations in tables describing biochemical
network models, kinetic parameters, and dynamic data:
\begin{enumerate}
\item General rules for the \textbf{structure of tables} and the \textbf{syntax} used in  table fields.
\item A list of defined \textbf{table types} for different kinds of
  information, each with possible \textbf{columns} with defined names
  and data types (see Table \ref{tab:tabletypes}; An overview of all
  predefined table types and their possible columns is given in the
  appendix).
\item A \textbf{syntax for biochemical element annotations} pointing to  databases
  or ontologies.
\item Rules for usage of \textbf{names}, \textbf{shortnames}, and \textbf{database identifiers}
  used for annotation.
\item \textbf{Naming rules for biochemical  quantities} (to specify the  quantities,
  physical units, and mathematical terms, like \defint{Mean} for mean values).
\item A syntax for \textbf{formulae describing reaction stoichiometries}. \la{simple kinetics, and regulation.}
\item The possibility to \textbf{extend the format} by declaring new column or table types.  
\end{enumerate}
While the SBtab rules can be used to structure various kinds of
information, the current version of SBtab is tailored for the
following kinds of data: (i) structure of biochemical network models;
(ii) biochemical quantities appearing in these models. \la{; (iii) omics
data.} This is reflected by the specific table types defined below.


This specification for SBtab (version 0.1.0) introduces the general SBtab rules as well as
specific formats and conventions for different use cases (see Section
\ref{chapter2}).  It defines the different types of SBtab tables (see
Section \ref{chapter3}) and explains the syntax of reaction \la{and
regulation} formulae in the SBtab format (see Section \ref{chapter4}).
Finally, the specification references the available online tools for
the handling of SBtab files (see Section \ref{chapter5}) and includes
an overview of all available SBtab table types in appendix
\ref{appendixA}. Appendix \ref{appendixB} lists controlled
vocabularies and database resources recommended to be used within
SBtab.

In the examples, elements of SBtab are highlighted in colours.  This
is just for convenience and is not a part of the SBtab format.
\tab{Table types} and \col{Column types} are defined by the SBtab
format are listed in Table
\ref{tab:tabletypes}.  \nick{Shortnames} are chosen arbitrarily; each
of them needs to be defined by a row in one of the tables.  Shortnames
have to be unique and consistent within a document, but can differ
between documents.  \defint{Reserved names} are predefined in SBtab
for recurrent expressions like ``mean value''.  \defext{Official
  names}, like for instance, the names used for databases, are defined
by some other authority.  \texttt{Free text} and other text including
database IDs, numerical values, mathematical brackets and operators.

\subsection{SBtab documents}

An SBtab document consists of several tables that refer to the same
model or related data sets. It may contain several tables of the same
table type, but the table names must be unique.  All tables must use a
common list of shortnames.  For instance, if a \tab{Compound}\, table
contains the column \col{!Compound}, the elements from this column
define the compound shortnames used in the other tables.  Several
tables in a document may have the same type, but not the same table
name (attribute \tab{TableName}).  If a document is stored in a
spreadsheet file, the tables should be called according to their table
name.  In delimited text tables (e.g.~.csv), there are two
possibilities to store a document: (i) can be stored in several files
with the filenames
\emph{basename}\texttt{\_}\emph{tablename}.\emph{extension};
(ii) A document containing several tables can be stored in a single 
table file. In this case, a declaration row (starting with ``!!'')  
has to precede each of the tables and tables are concatenated vertically.

To interpret a single table, other tables (e.g., describing
shortnames) may be required.  If a table does not require any other
tables, we call it ``semantically complete''. A document is
semantically complete if all names are defined, i.e., no additional
information is required to interpret its contents.  If a single table
or a document are not complete, the undefined names have to be known
by the software, and an exchange with other software tools is likely
to fail.  \la{Different IDs or names for the same elements have to
 agree, otherwise the parser will throw an error.} 
If a table or
document does not contain information implying that two elements
describe identical things, they should be treated as different.


\subsection{Names of biochemical elements}

\paragraph{Names and identifiers of model elements}
In the following, compounds, enzymes, genes, genetic regulators, and
compartments will be called ``biochemical entities''. ``Biochemical
elements'' comprises, in addition, reactions and biochemical
quantities.  Biochemical elements can be described by shortnames,
official names, or database identifiers (IDs). The shortnames have to
be declared within the SBtab document and have to satisfy certain
syntactic rules. Each table contains a column of the same name
containing the shortnames, i.e., arbitrary names used in a data set or
model. Shortnames must be unique, i.e., declared only once in a
document; they may not contain spaces or the special characters ``:'',
``.'' and may not start with the characters ``+'', ``-'', or ``\%''.
In columns containing database IDs, the database is specified in the
column name (\col{!MiriamID}::\emph{MiriamID}) by a name (to be used
in column names, IDs etc) and an URI.  We suggest to use preferably
the databases listed in the Miriam file (see Table
\ref{tab:databases}).  \la{Other databases can be declared in a table
  of type \tab{AnnotationResource}.} If an element is characterised
redundantly (e.g., reaction catalysed by an enzyme is given by both
shortname and ID), the information derived from the shortname (i.e.,
the ID listed in the \tab{Reaction} table) has higher priority.

\paragraph{Naming and specification of biological entities}
Tables of the types \tab{Compound}, \tab{Enzyme}, \tab{Gene},
\tab{Regulator}, or \tab{Compartment} are called ``entity tables''.
The identity of the entities can be
 declared by columns of type
\begin{itemize}
\item \col{!Name} contains  official names (good practice: use names from
  suggested databases).  To declare a database in which the name is
  used, it can also be written as \emph{DB}:\emph{name}. Several
  names can be listed in one field, separated by ``$|$''.
\item \col{!MiriamID}::\emph{MiriamID} contains IDs from a specified database.
  Annotations with database IDs follow the MIRIAM scheme (qualifier,
  resource, ID). Databases listed in the MIRIAM resources
  \cite{MIRIAM} are denoted in formuale by the name as given by the MIRIAM URN
  string.
\end{itemize}

\paragraph{Localised compounds}
If a compound, enzyme, or genetic regulator is localised in a
compartment, the corresponding localised entity can be denoted by
\emph{compound}\texttt{[}\emph{compartment}\texttt{]} with square
brackets, where \emph{compound} and \emph{compartment} refer to the
shortnames or IDs of the compound and the compartment used in the
model.  If a model contains several compartments, tools should treat
the first compartment in the \tab{Compartment} table as the standard
compartment to be used by default for all compounds that are not
explicitly assigned to compartments.


\subsection{Biochemical elements can be annotated with standard identifiers}
The rows of a table are annotated with annotations listed in special
identifier columns. \la{There are two types of identifier columns:

1.} A MiriamID column contains annotations of the same MIRIAM data type
(web resource), at most one annotation per element, and without
qualifiers.  The column item and the referenced ID are assumed to be
linked by an ``is'' relationship (and not, for instance, ``version
of'', which also exists in SBML annotations). A table can contain
several MiriamID columns, which must refer to different data
resources.

{\tt
  \begin{center}
    \begin{tabular}{|l|l|l|l|l|}
      \hline
      \tab{!!SBtab} & \tab{TableName='Compound'} & \tab{TableType='Compound'} & \\ \hline\hline
      \col{!Compound} & \col{!MiriamID}::\defext{urn:miriam:obo.chebi}& \col{!MiriamID}::\defext{urn:miriam:kegg.compound}&  ...  \\ 
      \hline
      \nick{water} & CHEBI:15377 & C00001  & ...\\\hline
      \nick{ATP}  & CHEBI:15422 & C00002  & ...\\\hline
      \nick{phosphate} & CHEBI:18367 & & ...\\\hline
    \end{tabular}
  \end{center}
}

To translate an element like \texttt{CHEBI:16865} into a valid MIRIAM
URN, the Miriam URI in the header (e.g.,
\defext{urn:miriam:obo.chebi}) is concatenated with the column item,
separated by a colon\footnote{The elements from the column have to be
translated into a URN-encoded form (as described in the URN
specification): for instance, the colon in the identifier
\texttt{CHEBI:16865} has to be replaced by the string
``\texttt{\%3A}'' to create the URN
\texttt{urn:miriam:obo.chebi:CHEBI\%3A16865}.}.


\subsection{Syntax for reaction \la{and regulation} formulae}
\label{chapter4}

Chemical reactions can be described by reaction formulae (specifying
the reactants, their stoichiometric coefficients, and possibly their
localisation).

\textbf{Reaction formulae} (column \col{!SumFormula} in table
  \tab{Reaction}).  The reaction arrow is denoted by \texttt{<=>}.
  Stoichiometric coefficients of 1 are omitted; general stoichiometric
  coefficients given by letters (e.g., \texttt{n}) are not allowed.
  Substrates and products are given by shortnames, which  must be defined 
  in a \tab{Compound} table.
  The order of substrates and the order of products are arbitrary;
  however, comparison of formulae is eased by using an
  alphabetical order. The localisation in compartments can be denoted as follows:

\begin{itemize}
\item Reaction in the default compartment: \texttt{\nick{A} + 2 \nick{B} <=> \nick{C} + \nick{D}}
\item Transport reaction: \texttt{\nick{A}[\nick{comp1}] + 2 \nick{B}[\nick{comp1}] <=> \nick{C}[\nick{comp2}] + \nick{D}[\nick{comp2}]}
\end{itemize}

In the example, \nick{A}, \nick{B}, \nick{C}, and \nick{D} are
compound shortnames, and \nick{comp1} and \nick{comp2} are compartment
shortnames.  


\subsection{Summary of SBtab rules}

SBtab implements the following conventions.
\begin{itemize}
\item \textbf{Shortnames} Model elements (e.g. compounds) are often
  referred to by shortnames which are
  defined in the corresponding table (e.g. \tab{Compound} for
  compounds) \la{or in the table \tab{Name} (for any kind of elements,
  properties, etc).}  Shortnames must be unique within an SBtab
  document.
  The first column of each table shares the name of the table type
  (e.g., column \col{Compound} in table type \tab{Compound}) and
  contains shortnames, which serve as primary keys for this table and must therefore be unique.
\item \textbf{Order of columns} While the allowed column types depend
  on the table type, their order is in most cases arbitrary. It is
  good practice to put the most important columns to the left.  
\item \textbf{Characters} The table fields contain only plain text.
  \la{Special symbols like Greek letters are ignored.} 
  The format is case-sensitive, but the choice of fonts (bold, italic) 
  does not play a role.
\item \textbf{Table types} The table types and their possible columns
  are defined in  appendix \ref{appendixA}.  
  Column names may not contain any special characters or white spaces (parsers should
  ignore these characters). 
\item \textbf{Comment lines} Table lines starting with a ``\%''
  character contain comments and are ignored during parsing.
\item \textbf{Comments and references} Additional information about
  table elements is stored in the optional columns \col{!Comment},
  \col{!Reference}, \col{!ReferencePubMed} and \col{!ReferenceDOI},
  which can appear in all tables.
\item \textbf{Unrecognised table or columns} Columns with unknown
  headers (not starting with \col{!}) and unrecognised header startgin with
  \col{!} may appear in SBtab tables, they can be used, but
  are not supported by the parser. The use of undefined columns is inadviseable.  
\item \textbf{Declaration line} The first line, starting with
  ``!!SBtab'' must declare at least the properties: \tab{TableType},
  \tab{TableName}, and possibly the properties \tab{Version}, \tab{Level} and \tab{Document}. 
  The entries can be separated by whitespaces or be given in separate fields.
\item \textbf{Identifiers}  Identifiers for compounds, compartments etc.~can be specified in
  columns with a header
  ``\emph{ElementType}\col{ MiriamID}::\emph{DB}'').  
\la{\item \textbf{Qualifiers} SBtab supports the qualifiers (like ``Is'',
  ``VersionOf'') defined in the MIRIAM resources \cite{bioqualifiers}.
  They can be used within table cells (in the syntax ``\emph{qualifier} \emph{MiriamID}''.}
\la{\item \textbf{Lists within table fields}  Some table fields may contain several entries of the same type
  (e.g., several alternative names for the same substance) separated by
  the ``$|$'' character.}
\item \textbf{Missing elements} If an element is missing, the table
  field is left empty.  Missing numerical values can be indicated by
  non-numerical elements like \texttt{?} or \texttt{na} (for ``not
  available'').  Mandatory fields must not by empty.
\item \textbf{Formulae} Reaction formulae \la{and formulae describing
 biochemical regulations} must be written in a special format explained
 below.  
\item \textbf{Physical units} In SBtab, it is recommended to use the
  units listed in the SBML specification (see
  {sbml.org/Documents/Specifications})\footnote{The following units
    are supported by SBML: {\tt ampere, gram, katal, metre, second,
      watt, becquerel, gray, kelvin, mole, siemens, weber, candela,
      henry, kilogram, newton, sievert, coulomb, hertz, Litre, ohm,
      steradian, dimensionless, item, lumen, pascal, tesla, farad,
      joule, Lux, radian}. Orders of magnitude can be denoted by {\tt
      k, M, c, m, mu, n, p, f} for Kilo, Mega, Centi, Milli, Micro,
    Nano, Pico, Femto.  If a parameter is dimensionless, it has to be
    annotated as {\tt dimensionless}.}.
\item \textbf{Reserved names} In the SBtab format, there are reserved
names for (i)  table types
(marked in \tab{red} in this document); (ii) column names (marked in
\col{blue})
(iii) types of biological elements (see table \ref{tab:objects}); and
(iv) types of biochemical quantities or mathematical terms (e.g.,
\defint{Mean}) for them (marked in \defint{green}, see Table
\ref{tab:quantities}), and physical units.
\end{itemize}

 \begin{table}[h!]
   \begin{center}
     \begin{tabular}{|l|l|l|}
       \hline
       Name & Contents & Usage  \\
       \hline
       \tab{Compound}   & Names, IDs, properties of compounds & model structure \\
       \tab{Enzyme}     & Names, properties of enzymes        & model structure \\
       \tab{Gene}       & Names, properties of genes          & model structure \\
       \tab{Regulator} & Names, properties of gene regulators & model structure \\
       \tab{Compartment} & Names and IDs of compartments      & model structure \\
       \tab{Reaction} &   Chemical reactions                  & model structure	\\
       \tab{Quantity} & Individual data for model parameters & quantitative data \\
       \tab{Relationship} & Relations between different compounds & model structure\\
	   \tab{Definition}   & Define custom column types, etc. & customise SBtab \\
       \hline
     \end{tabular}
   \end{center}
   \caption{Overview of table types in the SBtab format. \label{tab:tabletypes}}
 \end{table}


\section{Predefined table types in SBtab}
\label{chapter3}
SBtab has a number of predefined table types that can hold different
kinds of data. Each table type has a number of mandatory or optional
columns with specific properties. An overview is given in Table
\ref{tab:tabletypes}.  The table types \tab{Compound}, \tab{Enzyme},
\tab{Gene}, \tab{Regulator}, \tab{Compartment}, and \tab{Reaction}
describe model structures, the table types \tab{Quantity} and
\tab{Relationship} \la{, \tab{OmicsDataRow}, and \tab{OmicsDataColumn}} are
used for quantitative data.


\subsection{Tables for biochemical network structures}

As in the example in use case 1 (in the introduction section), biochemical 
networks consist of biochemical entities (e.g., metabolites or proteins) 
and reactions or interactions between them. The tables describing these 
entities (table types \tab{Reaction}, \tab{Compound}, \tab{Compartment}, 
\tab{Enzyme}, \tab{Regulator}, and \tab{Gene}) have to satisfy the following 
rules.

\begin{itemize}

\item \textbf{Entities} In tables describing biochemical entities
  (\tab{Compound}, \tab{Enzyme}, ...), each row has to contain (i) a
  shortname as the primary key (in the column \col{!Compound},
  \col{!Enzyme}, etc) and (ii) at least one entry specifying the
  entity, like \col{!Name} or \col{!MiriamID}:\emph{DB}.  If a column
  shares the type of the table (e.g., a Compound column in a Compound
  table), it can be considered a primary key, that is, its elements
  should be unique and it should appear as the first column in the
  table.  Optional columns - which may depend on the kinds of entities
  - are listed in Table \ref{tab:columnsentities}.

\item \textbf{Reactions}
A \tab{Reaction} table lists chemical reactions, possibly with
information about the corresponding enzymes, their kinetic laws, and
their genetic regulation. It must contain at least one of the
following columns: \col{!SumFormula},  \col{!SumFormulaJSON},\col{!MiriamID}:\emph{DB};  optional
columns are listed in Table \ref{tab:columnsreactions}. For an
example, see use case 1 in the introduction.

\item 
\textbf{Enzymes, genes, and regulators}
The connection between chemical reactions, the enzymes catalysing the
reactions, and the genes coding for the enzymes can be complicated,
but in many cases, there is a one-to-one relationship. In SBtab, there
are different ways to express this relationship.  Information about
enzymes or genes and their regulation can be stored in a
\tab{Reaction} table if there is a one-to-one relationship between
reactions, enzymes, and possibly genes.  Otherwise, it is stored in
separate tables \tab{Enzyme} and \tab{Gene} and the tables are
interlinked \emph{via} the columns
\col{!Enzyme} (in table \tab{Reaction}) and
\col{!Gene} (in table \tab{Enzyme}) or
\col{!TargetReaction} (in an \tab{Enzyme}
table) and \col{!GeneProduct} (in a \tab{Gene}
table). 

\end{itemize}

\paragraph{Conversion to SBML models} Reaction and Compound tables
can be translated into SBML (Systems Biology Markup Language).
Compound correspond to \texttt{species} in SBML. By default, the entries in
the \col{Compound} and \col{Reaction} columns are translated into \texttt{id}
attributes of the SBML elements. They can be overridden by SBML IDs
directly specified within SBtab in the columns \col{SBML:reaction:id},
\col{SBML:species:id}, \col{SBMLparameterID}, etc. 


\subsection{Tables for biochemical quantities}

Numerical data can be stored in two \la{four} different table types.
Tables of type \tab{Quantity} describe individual physical or
biochemical quantities, for instance, kinetic parameters in a network
model.  These quantities can be linked to one entity, one reaction or
enzyme, or both. 

Tables of type \tab{Quantity} describe single physical or biochemical
quantities (e.g., individual kinetic constants). 
A quantity is defined by a type, a unit, possibly
biochemical entities to which it refers, possibly a localisation, and
possibly experimental or physical conditions. The columns contain the
defining properties (e.g. unit, conditions, etc.)  and their values.
Quantities can refer to a compound, an enzyme or reaction, or a
combination of them. For instance, a concentration refers to a
substance, while a $k^{\rm M}$ value refers to a metabolite and an
enzyme. If there is a one-to-one relationship between reactions and
enzymes, the $k^{\rm M}$ value can also be assigned to a
compound/reaction pair or a compound/enzyme pair.  Let us consider
again use case 2:\\

{\tt \scriptsize
\begin{tabular}{|l|l|l|l|l|l|l|l|l|l|l|l|l|l|l|}
\hline
 \tab{!!SBtab}	& \tab{TableName='Quantity'} & \tab{TableType='Quantity'} & & & \\ \hline\hline
\tab{!Quantity} &  \col{!QuantityType} 	&  \col{!MiriamID:Reaction}::...\defext{reaction} & \col{!MiriamID:Compound}:...\defext{compound} &\defint{!Value} &\col{!Unit} \\ \hline\hline
\nick{kcrf\_R1} & \defext{substrate catalytic rate constant}   	&  R01061 &       & 200.0 & 1/s\\ \hline
\nick{keq\_R1} 	& \defext{equilibrium constant}      &  R01061 &        & 0.0984	& 1 \\ \hline
\nick{keq\_R2} 	& \defext{equilibrium constant}      &  R01061 &        & 0.156	 	& 1 \\ \hline
\nick{kmc\_R1\_C1} & \defext{Michaelis constant}     &  R01061 & C00003 & 0.96  	& mM\\ \hline
\nick{kic\_R1\_C1} & \defext{inhibition constant}    &  R01070 & C00111 & 0.13  	& mM\\ \hline
\nick{con\_C1} 	& \defext{concentration}             &         & C00118 & 0.203 	& mM\\ \hline
\end{tabular}
}

The abbreviation ``...'' in the URN is just for ease of reading; in an
SBtab file, URNs have to be written in full length.  To specify the
parameters of a model, we need to refer to \col{Reaction} and
\col{Compound} elements by shortnames rather
than by database IDs. In this form, the above example  could become\\

 {\tt \small
    \begin{tabular}{|l|l|l|l|l|l|}
\hline
    \tab{!!SBtab} & \tab{TableName='Quantity'} & \tab{TableType='Quantity'} & & & \\
      \hline\hline
      \col{!Quantity} &  \col{!MiriamID}::\defext{urn:miriam:obo.sbo}&  \col{!Reaction} & \col{!Compound} &\defint{!Value}  &\col{!Unit} \\ \hline
      \nick{kcrf\_R1} &  SBO:0000320 &  R1 &        & 200.0  & 1/s\\ \hline
      \nick{keq\_R1}   &  SBO:0000281 &  R1 &        & 0.0984 & 1 \\ \hline
      \nick{kmc\_R1\_C1} &  SBO:0000027 &  R1 & C1 & 0.96   & 1 \\ \hline
      \nick{kic\_R1\_C2} &  SBO:0000261 &  R1 & C2 & 0.13   & mM\\ \hline
      \nick{con\_C3}  &  SBO:0000196 &     & C3 & 0.203  & mM\\
      \hline                                                                               
\end{tabular}
}

This example also shows that quantity types can be specified in a
column \col{!QuantityType MiriamID}::\defext{urn:miriam:obo.sbo} by
identifiers from the Systems Biology Ontology (SBO).  The entries of
\tab{Quantity} tables can be inserted into SBML models and extracted
from them. By default, SBtab quantities referring to a reaction will
be translated into local parameters in SBML, while other quantities
will become global parameters. The element of the \col{Quantity}
column will be used as SBML element ID unless it is overridden by the
(optional) column \col{SBMLparameterID}. For the naming of kinetic
constants, see the conventions given in the supplementary file of
\cite{liuk:10}, supplementary material Table A.5.  Quantities that
describe initial species amounts, initial species concentrations, or
compartment sizes will be translated into the corresponding element
attributes.  A \tab{Quantity} table can also store state-dependent
quantities like concentrations, expression levels, or fluxes like in
the following example.

\begin{center} {\tt \small
    \begin{tabular}{|l|l|l|l|l|l|l|}
      \hline
    \tab{!!SBtab} & \tab{TableName='Quantity'} & \tab{TableType='Quantity'}& & & \\
\hline\hline
      \col{!Quantity}    &\col{!MiriamID::}... & !Condition & \col{!Compound} &\defint{!Value}  &\col{!Unit} \\ \hline
      \nick{con\_C1\_wt}    &  SBO:0000196 & wild type & C1 & 0.2  & mM\\ \hline
      \nick{con\_C2\_wt}    &  SBO:0000196 & wild type & C2 & 1    & mM\\\hline
      \nick{con\_C3\_wt}    &  SBO:0000196 & wild type & C3 & 0.1  & mM\\\hline
      \nick{con\_C1\_mu}    &  SBO:0000196 & mutant    & C1 & 0.1  & mM\\\hline
      \nick{con\_C2\_mu}    &  SBO:0000196 & mutant    & C2 & 0.5  & mM\\\hline
      \nick{con\_C3\_mu}    &  SBO:0000196 & mutant    & C3 & 0.1  & mM\\\hline
\end{tabular}
}
\end{center}


\section{SBtab online tools}
\label{chapter5}

To simplify and encourage the usage of SBtab, we provide several
online tools at \texttt{www.semanticsbml.org/SBtab/}.\coout{They are
  integrated in the collection of online services for SBML models,
  semanticSBML 2.0 (www.semanticsbml.org/semanticSBML), and comprise}
\begin{enumerate} 
\item \textbf{An online validator of SBtab files.} An online validator
  tool checks the validity of SBtab files (in .csv or Excel format)
  and instructs the user how to fix problems.  Validation is based on
  the known column types \la{(predefined declaration table accessible
    for each SBtab version and/or column table in the SBtab
    document)}.  The following validation rules are applied:
\begin{enumerate}
\item The table type and name must either be declared in the
  declaration line. The declaration line is checked for syntax.
  A declaration line and a table type definition is mandatory for the 
  functionality of SBtab. If the attribute \tab{TableName} is missing, it will
  set automatically using the \tab{TableType} followed by a number.
\item Check the first column, is it consistent with the table type?
  Are all columns declared for this table type? If not, issue a warning and print 
  a list of all possible column types.
\item Check file for compatibility with tablib. 
\end{enumerate}

\item \textbf{An online parser for SBtab.} The SBtab parser uses tablib \footnote{http://docs.python-tablib.org/en/latest/} to import the SBtab file.\\
The parser itself has different functions to edit the data and to use it in 
Python directly:
\begin{enumerate}
\item Read table information (type, name, etc...) from the table.
\item Addition of rows and columns to the SBtab table.
\item Editing and export of the table content in rows, columns and single entries. 
The export as Python dictionary is also provided to ensure the best possible access to the data.
\item Switching of columns and rows (transpose) of the table. As some data is
stored more convenient in a transposed spread sheet, some tables need to be 
transposed to have better access to its content. 
\end{enumerate}

\item \textbf{An online converter of SBtab files into SBML files.}  An
  SBtab document can be converted into a valid SBML model.  The SBtab
  file needs to contain a table of the type \tab{Reaction}; other tables
  can provide additional information for the SBML model (e.g.
  \tab{Compound}, \tab{Enzyme}, or \tab{Compartment}).
\item \textbf{An online converter of SBML files into SBtab files.} A
  model in SBML format can be converted into an SBtab document
  containing various types of tables (e.g. \tab{Reaction}, \tab{Compound},
  \tab{Compartment}, \tab{Enzyme}, and possibly \tab{Quantity}).
\end{enumerate}


\section*{Acknowledgements}
The authors thank Hans-Michael Kaltenbach, Dirk Wiesenthal, Jannis
Uhlendorf, Anne Goelzer, J\"org B\"uscher, Avi Flamholz and Phillipp Schmidt for
contributing to this proposal. This work was funded within the
BaSysBio project.

\bibliographystyle{unsrt}
\bibliography{/home/wolfram/latex/bibtex/biology}

\clearpage

\begin{appendix}

\section{Overview of table types}
\label{appendixA}

\coout{add SBMLSpeciesID wherever needed}
\coout{mark columns as * mandatory x recommended o optional . none}

\begin{table}[h!]
\textbf{All table types}\\
\begin{tabular}{|l|l|l|l|}
\hline 
Name & Format & Type & Content \\
\hline
  \col{!Description}    & text & string & Description of the row element\\
  \col{!Comment}        & text & string & Comment\\
  \col{!ReferenceName}  & text & string & Reference title, authors, etc.~(as free text)\\
  \col{!ReferencePubMed}& text & string & Reference PubMed ID\\
  \col{!ReferenceDOI}   & text & string & Reference DOI\\
  \hline 
\end{tabular}
\caption{Columns that can appear in all tables}
\label{tab:columnsalltables}
\end{table}

\begin{table}[h!]
{\small\textbf{All entity and reaction tables} \\
\begin{tabular}{|l|l|l|l|}
\hline
Name & Type & Format & Content \\
\hline
  \col{!Name}       				& text 			& string 	& Entity name \\
  \col{!MiriamID}::\emph{MiriamURN} & resource ID 	& string	& Entity ID \\
  \col{!MiriamAnnotations} 			& annotation 	& JSON string 	& Entity ID \\
  \col{!Type}					 	& shortname 	& string	& Biochemical type of entity (examples see Table \ref{tab:objects})\\
\hline
\end{tabular}
\caption{Columns that can appear in all entity and reaction tables.}
\label{tab:columnsalltables}
}
\end{table}

\begin{table}[h!]
\tab{Compound}\\
\begin{tabular}{|l|l|l|l|}
\hline
Name & Type & Format & Content \\
\hline
  \col{!Compound}   			& shortname 		& string 		& Compound shortname\\
  \col{!SBML:species:id}   		& SBML element ID	& string		& SBML Species ID of the entity \\
  \col{!SBML:speciestype:id}	& SBML element ID 	& string		& SBML SpeciesType ID of the entity \\
  \col{!Location} 				& shortname 		& string 		& Compartment for localised entities\\
  \col{!State}    				& shortname 		& string		& State of the entity \\
  \col{!CompoundSumFormula}     & text 				& string 		& Chemical sum formula \\
  \col{!StructureFormula} 		& text 				& string 		& Chemical structure formula \\
  \col{!Charge}  				& number 			& integer		& Electrical charge number \\
  \col{!Mass}					& number 			& float 		& Molecular mass \\
  \col{!Unit} 					& text 				& string 		& Physical unit\\
  \col{!IsConstant}       		& Boolean 			& True, False 	& Substance with fixed concentrations\\
  \col{!EnzymeRole}  	  		& shortname 		& string		& Enzymatic activity \\
  \col{!RegulatorRole}  	 	& shortname 		& string		& Regulatory activity \\
\hline
\end{tabular}
\caption{Columns that can appear in \tab{Compound} tables}
\label{tab:columnsalltables}
\end{table}

\begin{table}[h!]
\tab{Enzyme}\\
\begin{tabular}{|l|l|l|l|}
  \hline
  Name & Type & Format & Content \\
  \hline
  \col{!Enzyme} 			& shortname & string & Enzyme shortname\\
  \col{!CatalysedReaction}	& shortname & string & Catalysed reaction \\
  \col{!KineticLaw} 		& shortname & string & Catalysed reaction \\
  \col{!Gene}  				& shortname & string & Gene coding for enzyme\\
  \hline
\end{tabular}
\caption{Columns that can appear in \tab{Enzyme} tables}
\label{tab:columnsentities1}
\end{table}

\begin{table}
{\small
  \tab{Gene} \\ 
\begin{tabular}{|l|l|l|l|}
  \hline
  Name & Type & Format & Content \\
\hline
  \col{!Gene} 		& shortname 	& string	& Gene shortname\\
  \col{!GeneLocus} 	& string 		& string 	& Locus name\\
  \col{!GeneProduct}& shortname 		& string & Gene product \\
  \col{!GeneProduct SBML:species:id} 	& SBML element ID 	& string & SBML ID of protein \\
  \col{!Operon}  	& shortname 		& string & Operon in which gene is located\\
\hline
\end{tabular}}
\caption{Columns that can appear in \tab{Gene} tables}
\label{tab:columnsalltables}
\end{table}

\begin{table}[h!]
  \tab{Regulator} \\
\begin{tabular}{|l|l|l|l|}
 \hline
  Name & Type & Format & Content \\
  \hline
  \col{!Regulator} 		& shortname & string & Regulator shortname\\
  \col{!State} 			& shortname & string & State of the regulator \\
  \col{!TargetGene}  	& shortname & string & Target gene\\
  \col{!TargetOperon}  	& shortname & string & Target operon\\
  \col{!TargetPromoter} & shortname & string & Target promoter\\
\hline
\end{tabular}
\caption{Columns that can appear in \tab{Regulator} tables}
\label{tab:columnsalltables}
\end{table}

\begin{table}[h!]
  \tab{Compartment} \\
  \begin{tabular}{|l|l|l|l|}
    \hline
    Name & Type & Format & Content \\
    \hline
    \col{!Compartment} 				& shortname 		& string & Compartment shortname\\
    \col{!SBML:compartment:id}   	& SBML element ID 	& string & SBML Compartment ID \\
    \col{!OuterCompartment} 		& shortname 		& string & Surrounding compartment \\
    \col{!SBML:OuterCompartment:compartment:id} & SBML element ID & string & Surrounding compartment \\
    \col{!Size} 					& number 			& float  & Compartment size \\
    \col{!Unit} 					& text 				& string & Physical unit\\
    \hline
  \end{tabular}
\caption{Columns that can appear in \tab{Compartment} tables}
\label{tab:columnsalltables}
\end{table}

\begin{table}[h!]
\label{tab:columnsentities}
\end{table}

\begin{table}
  \tab{Reaction} \\
\begin{tabular}{|l|l|l|l|}
  \hline
  Name & Type & Format & Content \\
  \hline
  \col{!Reaction}       	& shortname 		& string & Reaction shortname \\
  \col{!SBML:reaction:id}   & SBML element ID 	& string & SBML Reaction ID \\
  \col{!SumFormula} 		& SumFormula formula & string & Reaction sum formula \\
  \hline
  \col{!Location}   		& shortname & string & Compartment for localised reaction\\
  \col{!Enzyme}  			& shortname & Fstring & Enzyme catalysing the reaction \\
  \col{!Model}   			& text 		& string & Model(s) in which reaction is involved \\
  \col{!Pathway}  			& text 		& string & Pathway(s) in which reaction is involved \\
  \col{!SubreactionOf} 		& shortname & string & Mark as subreaction of a (lumped) reaction\\
  \hline
  \col{!IsComplete}    		& Boolean & True, False & Reaction formula includes all cofactors etc\\
  \col{!IsReversible}  		& Boolean & True, False & Reaction should be treated as irreversible\\
  \col{!IsInEquilibrium}    & Boolean & True, False & Reaction approximately in equilibrium\\
  \col{!IsExchangeReaction} & Boolean & True, False & Some reactants are left out \\ 
  \hline
  \col{!Flux} 				& number    & float & Metabolic flux through the reaction  \\
  \col{!IsNonEnzymatic} 	& Boolean   & True, False & Non-catalysed reaction \\
  \col{!KineticLaw}   		& shortname & string 	& see table type \tab{Enzyme} \\
  \col{!Gene} 				& shortname & string 	& see table type \tab{Enzyme} \\
  \col{!Operon}  			& shortname & string 	& see table type \tab{Gene}\\
  \hline
  \col{!Enzyme SBML:species:id}    		& SBML element  ID &string & SBML ID of enzyme   \\
  \col{!Enzyme SBML:parameter:id}    	& SBML element  ID &string & SBML ID of enzyme   \\
  \col{!BuildReaction}$^{M}$         	& Boolean &  True, False &  Consider the reaction in SBML model\\
  \col{!BuildEnzyme}$^{M}$           	& Boolean &  True, False & Include enzyme  in SBML model\\
  \col{!BuildEnzymeProduction}$^{M}$ 	& Boolean &  True, False & Describe enzyme production in SBML model\\
  \hline
\end{tabular}

\caption{Columns in tables of type \tab{Reaction}.  The lower section lists, again,  column types from Table \ref{tab:columnsentities}.}
\label{tab:columnsreactions}
\end{table}

\begin{table}
  \tab{Relationship} \\
\begin{tabular}{|l|l|l|l|}
  \hline
  Name & Type & Format & Content \\
  \hline
  \col{!Relationship}   & shortname & string 		& Type of quantitative relationship\\
  \col{!From}  			& shortname & string 		& Element at beginning of arrow \\
  \col{!To}    			& shortname & string 		& Element at arrowhead\\
  \col{!IsSymmetric}    & Boolean 	& True, False 	& Flag indicating non-symmetric relationships \\ 
  \col{!Value}:\emph{QuantityType} & number  & float & Numerical value assigned to the relationship \\ 
  \hline
\end{tabular}
\caption{Columns that can appear in \tab{Relationship} tables.}
\label{tab:columnsrelations}
\end{table}

\begin{table}
\tab{Quantity} \\
\begin{tabular}{|l|l|l|l|}
\hline
  Name & Type & Format & Content \\
  \hline
  \col{!Quantity}  		& shortname & string & Quantity / SBML parameter ID \\
  \col{!QuantityType}   & shortname & string & Quantity type (e.g., from SBO) \\
  \emph{ValueType}  	& ValueType & string & Mathematical Term from table \ref{tab:quantityvaluetype} (Mean, Std,...)\\
  \col{!SBML:parameter:id}   	  & SBML element ID &  string & Parameter ID in SBML file \\
  \col{!Unit} 			& text 		& string & Physical unit\\
  \col{!Scale} 			& text 		& string & Scale (e.g., logarithm, see Table  \ref{tab:scale}) \\
  \col{!Condition} 		& text 		& string & experimental condition name (free text)\\
  \col{!pH} 			& number 	& float  & pH value in measurement\\
  \col{!Temperature} 	& number 	& float  & Temperature in measurement\\
  \col{!Location}     	& shortname & string & Compartment  \\
  \col{!SBML:compartment:id}      &  SBML element ID & string& SBML ID of compartment`  \\
  \col{!Compound}      	& shortname & string & Related compound  \\
  \col{!MiriamID:Compound}::\emph{MiriamURN} & resource ID &string & Compound ID \\
  \col{!SBML:Compound:species:id} &  SBML element ID & string & SBML ID of compound  \\
  \col{!Reaction}   	& shortname & string & Related reaction  \\
  \col{!MiriamID:Reaction}::\emph{MiriamURN} & resource ID & string & Reaction ID \\
  \col{!SBML:reaction:id}         & SBML element ID & string & SBML ID of reaction  \\
  \col{!Enzyme}    		& shortname & string & Related enzyme  \\
  \col{!SBML:Enzyme:species:id}   & SBML element  ID & string & SBML ID of enzyme  \\
  \col{!SBML:Enzyme:parameter:id} & SBML element  ID & string& SBML ID of enzyme  \\
  \col{!MiriamID:Enzyme}::\emph{MiriamURN} & resource ID & string & Enzyme ID \\
  \col{!Gene}      		& shortname & string & Related gene  \\
  \col{!Organism}   	& shortname & string & Organism \\
\hline
\end{tabular}
\caption{Columns for numerical values and experimental conditions in tables of type 
\tab{Quantity}.}
\label{tab:columnsquantity}
\end{table}

\clearpage

\section{Controlled vocabularies and database resources}
\label{appendixB}


\begin{table}[h!]
  \begin{center}
    \begin{tabular}{|l|l|l|l|}
      \hline
      ValueType       & Type  & Format &  Meaning\\
      \hline                            
      \defint{Value} & number & float & Simple value  \\
      \defint{Mean}  & number & float & Algebraic mean  \\
      \defint{Std}   & number & float $>0$ & Standard deviation  \\
      \defint{Min}   & number & float & Lower bound \\
      \defint{Max}   & number & float & Upper bound  \\
      \defint{Median}& number & float & Median  \\
      \defint{GeometricMean}& number & float & Geometric mean  \\
      \defint{Sign}  & sign & +,-,0 & Sign  \\
      \defint{ProbDist} & Free text & string & Probability distribution: \\
      \hline
    \end{tabular}\hspace{5mm}
    \begin{tabular}{|l|l|l|l|}
      \hline
      Scale  &  Meaning\\
      \hline              
      \defint{Lin}   & Linear scale (no transformation)  \\
      \defint{Ln}    & Natural logarithm  \\
      \defint{Log2}  & Dual logarithm  \\
      \defint{Log10} & Decadic logarithm \\
      \hline
    \end{tabular}
  \end{center}
  \caption{Left: Quantity mathematical terms recommended for use in SBtab. 
    Names of probability distributions can be, for instance, 
    \texttt{Normal, Uniform, LogNormal}. Right:Mathematical scales used for data.}
\label{tab:quantityvaluetype}
\end{table}


\begin{table}[h!]
  \begin{center}
{\small
    \begin{tabular}{|l|l|l|l|}
\hline
  Database & MIRIAM URN           & Contents & URI \\
\hline
  SBO                 &  \defext{urn:miriam:obo.sbo}       & Quantities, rate laws               & www.ebi.ac.uk/sbo/ \\
  CheBI               &  \defext{urn:miriam:obo.chebi}     & Metabolites    & www.ebi.ac.uk/chebi/ \\
  Enzyme nomenclature &  \defext{ec-code}       & Enzymes        & www.ebi.ac.uk/IntEnz/ \\ 
  KEGG Compound       &  \defext{urn:miriam:kegg.compound} & Compounds & www.genome.jp/KEGG/ \\
  KEGG Reaction       &  \defext{urn:miriam:kegg.reaction} & Reactions & www.genome.jp/KEGG/ \\
  UniProt             &  \defext{uniprot}       & Proteins       & www.uniprot.org/  \\
  Gene Ontology       &  \defext{urn:miriam:obo.go}        & Compartments   & www.geneontology.org/ \\
  Taxonomy            &  \defext{taxonomy}      &  Organisms     & www.ncbi.nlm.nih.gov/Taxonomy/ \\
  SGD                 &  \defext{sgd}           & Yeast proteins & www.yeastgenome.org/ \\
\hline
\end{tabular}}
\end{center}
\caption{A selection of databases to be used in SBtab. For the complete list, 
  see the  MIRIAM resources. \la{Additional resources can be defined in a \tab{AnnotationResource} table.}}
 \label{tab:databases}
\end{table}


\begin{table}[h!]
  \begin{center} \small
    \begin{tabular}{|l|l|l|l|l|}
      \hline
      Name & SBO term & Default unit & Entities & Prefix \\
      \hline
      \defext{standard Gibbs energy of formation} &  SBO:0000582  & kJ/mol & Compound & \texttt{scp}\\
      \defext{equilibrium constant}       &  SBO:0000281  &           & Reaction & \texttt{keq} \\
      \defext{forward maximal velocity}   &  SBO:0000324            & mMol/s & Enzymatic Reaction & \texttt{vmaf} \\
      \defext{reverse maximal velocity}  & SBO:0000325  & mMol/s& Enzymatic Reaction & \texttt{vmar} \\
      \defext{substrate catalytic rate constant}      & SBO:0000321     & 1/s       & Enzymatic Reaction & \texttt{kcrf} \\
      \defext{product catalytic rate constant}     & SBO:0000320 & 1/s      & Enzymatic Reaction & \texttt{kcrr} \\
      \defext{Michaelis constant}         &  SBO:0000027  & mM    & Enzyme, Compound & \texttt{kmc} \\
      \defext{inhibitory constant}        &  SBO:0000261 & mM       & Enzyme, Compound & \texttt{kic} \\
      \defext{activition constant}        &  SBO:0000363  & mM       & Enzyme, Compound & \texttt{kac} \\
      \defext{Hill constant}              &  SBO:0000190  & dimensionless   & Compound, Reaction & \texttt{hco} \\ 
      \hline                                             
      \defext{pH}                           &  SBO:0000304  & dimensionless & Location  &  \\
      \defext{concentration}                &  SBO:0000196  & mM       & Compound  & \texttt{con} \\
      \defext{biochemical potential}           & SBO:0000303 & kJ/mol   & Compound &  \\
      \defext{standard biochemical potential}           & SBO:0000463 & kJ/mol   & Compound & \texttt{scp} \\
      \hline
    \end{tabular}
  \end{center}
  \caption{A selection of quantity types to be used in SBtab in table types 
    \tab{Quantity}.
    The unit of  equilibrium constants depends on the reaction stoichiometry. 
    More quantities can be found in the Systems Biology Ontology.}
\label{tab:quantities}
\end{table}

\begin{table}[h!]
  \begin{center}
{\small
  \begin{tabular}{|l|l|l|}
      \hline
      \textbf{Physical entity types} &             \\ \hline
      \defext{protein complex       }& SBO:0000297 \\
      \defext{messenger RNA         }& SBO:0000278 \\
      \defext{ribonucleic acid      }& SBO:0000250 \\
      \defext{deoxyribonucleic acid} & SBO:0000251 \\
      \defext{polypeptide chain     }& SBO:0000252 \\
      \defext{polysaccharide        }& SBO:0000249 \\
      \defext{metabolite            }& SBO:0000299 \\
      \defext{macromolecular complex}& SBO:0000296 \\
	  \hline
    \end{tabular}\hspace{5mm}
  \begin{tabular}{|l|l|l|}
      \hline
      \textbf{Compartments} &             \\ \hline
      \defext{cell }  & GO:0005623  \\
      \defext{extracellular space }  & GO:0005615  \\
      \defext{membrane            }  & GO:0001602  \\
      \defext{cytosol             }  & GO:0005829  \\
      \defext{nucleus             }  & GO:0005634  \\
      \defext{mitochondrion       }  & GO:0005739  \\  
      \hline
\hline
    \end{tabular}
}
  \end{center}
\caption{Examples of biochemical entity types (with Systems Biology Ontology identifiers) and 
cell compartments (with Gene Ontology identifiers).}
\label{tab:objects}
\end{table}
 
\end{appendix}

\end{document}
